% Options for packages loaded elsewhere
\PassOptionsToPackage{unicode}{hyperref}
\PassOptionsToPackage{hyphens}{url}
%
\documentclass[
]{article}
\usepackage{amsmath,amssymb}
\usepackage{lmodern}
\usepackage{iftex}
\ifPDFTeX
  \usepackage[T1]{fontenc}
  \usepackage[utf8]{inputenc}
  \usepackage{textcomp} % provide euro and other symbols
\else % if luatex or xetex
  \usepackage{unicode-math}
  \defaultfontfeatures{Scale=MatchLowercase}
  \defaultfontfeatures[\rmfamily]{Ligatures=TeX,Scale=1}
\fi
% Use upquote if available, for straight quotes in verbatim environments
\IfFileExists{upquote.sty}{\usepackage{upquote}}{}
\IfFileExists{microtype.sty}{% use microtype if available
  \usepackage[]{microtype}
  \UseMicrotypeSet[protrusion]{basicmath} % disable protrusion for tt fonts
}{}
\makeatletter
\@ifundefined{KOMAClassName}{% if non-KOMA class
  \IfFileExists{parskip.sty}{%
    \usepackage{parskip}
  }{% else
    \setlength{\parindent}{0pt}
    \setlength{\parskip}{6pt plus 2pt minus 1pt}}
}{% if KOMA class
  \KOMAoptions{parskip=half}}
\makeatother
\usepackage{xcolor}
\IfFileExists{xurl.sty}{\usepackage{xurl}}{} % add URL line breaks if available
\IfFileExists{bookmark.sty}{\usepackage{bookmark}}{\usepackage{hyperref}}
\hypersetup{
  hidelinks,
  pdfcreator={LaTeX via pandoc}}
\urlstyle{same} % disable monospaced font for URLs
\usepackage[margin=1in]{geometry}
\usepackage{longtable,booktabs,array}
\usepackage{calc} % for calculating minipage widths
% Correct order of tables after \paragraph or \subparagraph
\usepackage{etoolbox}
\makeatletter
\patchcmd\longtable{\par}{\if@noskipsec\mbox{}\fi\par}{}{}
\makeatother
% Allow footnotes in longtable head/foot
\IfFileExists{footnotehyper.sty}{\usepackage{footnotehyper}}{\usepackage{footnote}}
\makesavenoteenv{longtable}
\usepackage{graphicx}
\makeatletter
\def\maxwidth{\ifdim\Gin@nat@width>\linewidth\linewidth\else\Gin@nat@width\fi}
\def\maxheight{\ifdim\Gin@nat@height>\textheight\textheight\else\Gin@nat@height\fi}
\makeatother
% Scale images if necessary, so that they will not overflow the page
% margins by default, and it is still possible to overwrite the defaults
% using explicit options in \includegraphics[width, height, ...]{}
\setkeys{Gin}{width=\maxwidth,height=\maxheight,keepaspectratio}
% Set default figure placement to htbp
\makeatletter
\def\fps@figure{htbp}
\makeatother
\setlength{\emergencystretch}{3em} % prevent overfull lines
\providecommand{\tightlist}{%
  \setlength{\itemsep}{0pt}\setlength{\parskip}{0pt}}
\setcounter{secnumdepth}{5}
\usepackage{booktabs}
\usepackage{longtable}
\usepackage{array}
\usepackage{multirow}
\usepackage{wrapfig}
\usepackage{float}
\usepackage{colortbl}
\usepackage{pdflscape}
\usepackage{tabu}
\usepackage{threeparttable}
\usepackage{threeparttablex}
\usepackage[normalem]{ulem}
\usepackage{makecell}
\usepackage{xcolor}
\ifLuaTeX
  \usepackage{selnolig}  % disable illegal ligatures
\fi

\author{}
\date{\vspace{-2.5em}}

\begin{document}

\hypertarget{model-of-interest-and-data}{%
\subsection{Model of Interest and Data}\label{model-of-interest-and-data}}

\hypertarget{MOI}{%
\subsubsection{Model of Interest}\label{MOI}}

After controlling for the linear effects of dementia status (1 = diagnosed with MCI or dementia, 0 = otherwise), sex (1 = female, 0 = male), race (1 = white, 0 = other), age (mean centered), education (mean centered), and APOE e4 allele status (1 = has at least 1 APOE e4 allele, 0 = otherwise), we wish to compare latent cognition processes between those who have transitioned to MCI or dementia during follow-up (transitioners) and those who have not (non-transitioners). Previous research suggests that the effect of APOE e4 status differs between males and females, therefore an interaction between e4 status and sex is included in the model. To measure the linear effect of the dependent variables on the cognition tests trajectory, all dependent variables are put into the model as an interaction with time. In the group that have transitioned to MCI or dementia, we include a linear spline for when the participant transitions from cognitively normal to MCI or dementia.

\hypertarget{DAT}{%
\subsubsection{Data}\label{DAT}}

Data collected by the National Alzheimer's Coordinating Center (NACC) Uniform Data Set Version 2.0 (UDS, September 2014) was used to test the proposed MLLT model validity. The data analysis utilized two different groups in the NACC. Criteria for entry into the the first group requires Alzheimer's disease participants to have transitioned from cognitively normal to mild cognitive impairment (MCI) or Dementia during the NACC follow-up period. The second group are those who remained cognitively normal throughout the follow-up period. For the multivariate modeling of different cognition domains, we used the following neuropsychological outcomes: logic memory, memory units, digit forward, digit backward, trails A, trails B, WAIS, boston naming, and animals naming. Predictors of interest include: time, race white, sex, apoe e4 status, education, time of transition from cognitively normal to mild cognitive impairment (MCI) or dementia, and age. Inclusion into the analysis requires non-missingness for covariates and cognitive outcome at a given time point, as well as each participant needing 1 or more complete observation.

That are several cohort demographic differences in those who transitioned to MCI or dementia and those with no dementia diagnosis. As expected, transitioners are typically older at enrollment (78.45 years vs.~70.83 years), have more return visits into their older age (5.13 visits vs.~4.46 visits), and have higher rates of having an APOE e4 allele (38.26\% vs.~26.99\%). To assure latent cognition process correlations between dementia transitioners and non-dementia transitioners only differ due to dementia status, we conduct a sensitivity analysis with a more homogenous comparison group. To do this, we carry out 1-to-1 propensity matching with a caliper of 0.1 using all described linear predictors. Of the 737 transitioners, 734 matched to non-transitioners. In the transitioner and non-transitioner groups there is more similarity in enrollment age (78.40 years vs 78.44), return visits (5.13 visits vs.~5.25 visits), and e4 allele status (38.15\% vs.~38.15\%).

\begin{table}

\caption{\label{tab:unnamed-chunk-4}Demographic description for matched and unmatched datasets.}
\centering
\begin{tabular}[t]{lllll}
\toprule
\multicolumn{1}{c}{ } & \multicolumn{2}{c}{Non-Matched} & \multicolumn{2}{c}{Matched} \\
\cmidrule(l{3pt}r{3pt}){2-3} \cmidrule(l{3pt}r{3pt}){4-5}
  & Dementia & Non-Dementia & Dementia & Non-Dementia\\
\midrule
N participants & 737 & 6136 & 734 & 734\\
Return visits & 5.13 (2.06) & 4.46 (2.24) & 5.13 (2.05) & 5.25 (2.33)\\
APOE e4 allele & 38.26\% & 26.99\% & 38.15\% & 35.15\%\\
Sex (female) & 62.69\% & 67.42\% & 62.53\% & 62.13\%\\
Race (white) & 87.65\% & 83.34\% & 87.6\% & 87.87\%\\
\addlinespace
Age & 78.45 (8.25) & 70.83 (10.18) & 78.4 (8.23) & 78.44 (8.45)\\
Education yrs & 16.14 (8.11) & 16.13 (5.95) & 16.02 (7.53) & 15.83 (4.28)\\
Follow-up trs & 5.02 (2.17) & 4.18 (2.48) & 5.01 (2.16) & 5.14 (2.5)\\
Time of decline
(tears since baseline) & 3.54 (2.11) & - & 3.53 (2.1) & -\\
\bottomrule
\end{tabular}
\end{table}

\end{document}
